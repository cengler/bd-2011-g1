\documentclass[a4paper, 10pt, notitlepage]{article}

\usepackage{moreverb} %para importar codigo

\usepackage{pepotina} %paquete personal para la caratula del DC

\usepackage[spanish,activeacute]{babel}
\usepackage{babel} %paquete de idioma

\usepackage[latin1]{inputenc}

\usepackage[normalem]{ulem}
\usepackage{ulem}

%\usepackage{color}

\usepackage{hyperref}
%\usepackage[all]{hypcap}

\usepackage{caeycaeBD}

\usepackage{fancyhdr} %linea sup con comentarios

\usepackage{lscape} %para hoja apaisada

\usepackage{framed} %para crear cajas de texto

\usepackage{lastpage} %ultima pagina

%\usepackage{pstricks}
%\usepackage{uml} %UML

\usepackage{listings}
%\lstset{
%  breaklines=true,                                     % line wrapping on
%  language=ocl,
%  frame=ltrb,
%  framesep=5pt,
%  basicstyle=\normalsize,
%  keywordstyle=\ttfamily\color{OliveGreen},
%  identifierstyle=\ttfamily\color{CadetBlue}\bfseries,
%  commentstyle=\color{Brown},
%  stringstyle=\ttfamily,
%  showstringspaces=ture
%}

\addtolength{\topmargin}{-50pt} 
\addtolength{\textwidth}{145pt}
\addtolength{\textheight}{120pt}
\addtolength{\oddsidemargin}{-70pt}

%\newcommand{\minix}{\textsl{minix }}

%%% Encabezado y pie de p'agina
\pagestyle{fancy}
\fancyhead[LO]{Base de Datos}
\fancyhead[C]{}
\fancyhead[RO]{P\'agina \thepage\ de \pageref{LastPage}}
\renewcommand{\headrulewidth}{0.4pt}
\fancyfoot{}

\newcommand{\FALTA}{{\bf FALTA FALTA FALTA FALTA FALTA FALTA FALTA FALTA }}
\def\var#1{\textsl{#1}}


\begin{document}

\universidad{Universidad de Buenos Aires}
\facultad{Facultad de Ciencias Exactas y Naturales}
\departamento{Departamento de Computaci�n}
\materia{Base de Datos}
\resumen{El presente trabajo pr�ctico consiste en el desarrollo de una base de datos para una empresa de transporte. Se realiz� un diagrama entidad relaci�n de los requerimientos. A partir del MER se construy� el modelo relacional. Finalmente se llev� lo anterior a un dise�o f�sico utilizando el motor de base de datos SQL server 2005.}
\keys{MER, DER, MR, SQL, Base de Datos, Base de Datos Relacional}
\titulo{Tp1}
\subtitulo{Tp 1: Empresa de Transportes}
\grupo{N�mero de Grupo: 1}
\fecha{2do Cuatrimestre 2011}
\footspace{1cm}
\integrante{Piotrkowski, Kevin}{204/06}{marcoskevin@gmail.com}
\integrante{Rugnone, Mart�n}{665/04}{mrugnone@hotmail.com}
\integrante{Alvarez, Mar�a de los Angeles}{264/05}{mdelosaalvarez@hotmail.com}
\integrante{Engler, Christian}{314/05}{caeycae@gmail.com}

\maketitle{}

\tableofcontents

\newpage


\section{Introducci�n}
Este trabajo presenta la solucion al Trabajo Practico ``Bases de Datos, Segundo Cuatrimestre de 2011''\footnote{Enunciado del Trabajo Practico \textit{TP Primera Parte-2c2011.pdf}}. En este documento presentamos las hipotesis, decisiones de dise�o, y dise�o fisico de la soluci�n propuesta.

La solucion se encuenta desarrollada con las siguientes metodologias y herramientas.

\begin{itemize}
	\item Modelo de Entidad Relaci�n y Modelo Relacional derivado.
	\item Detalle de los supuestos asumidos para la resoluci�n del problema. (Hipotesis)
	\item Dise�o f�sico correspondiente a la soluci�n implementada. (En SQL Server 2005)
	\item Restricciones adicionales al modelo
	\item C�digo correspondiente a los stored procedures, triggers que se hayan implementado en la soluci�n.
\end{itemize}



\section{Hip�tesis}
\begin{enumerate}
	\item Una ruta es un camino posible para ir desde un punto a otro.
	\item Un recorrido no puede comenzar y finalizar en el mismo lugar.
	\item Con lo que respecta a las condiciones del a�o seg�n el periodo, este �ltimo se tom� como las estaciones del a�o. Es decir: invierno, oto�o, primavera y verano.
	\item No nos interesa discriminar el nombre del apellido de un chofer, pues no se cuenta con ning�n requerimiento para realizar dicha separaci�n(ej: alguna consulta para filtrar viajes por apellido del chofer).		
	\item Consideramos que \textbf{estado} del veh�culo se refiere a una descripci�n general del mismo(ej:buen estado, averiado, abollado,...) en vez de referirse al discriminante que separa a los veh�culos en: ``en uso'' y ``en reparaci�n''.
\end{enumerate}


\newpage

\section{MER}

\subsection{Decisiones de Dise�o}

\begin{enumerate}
	\item Ruta es d�bil en relaci�n a un recorrido porque s�lo sirve como una forma de ir desde el origen del recorrido a su destino.
	\item Contingencia es una entidad d�bil porque puede haber una cantidad arbitraria de ellas por cada viaje realizado(en contraposici�n a una lista de contingencias predefinidas ej:choque, bloqueo de ruta, control policial, etc) y las mismas no tienen sentido sin su viaje relacionado.
	\item La ruta tiene asociado un clima por cada periodo del a�o. Como se menciono en las hipotesis el periodo son las estaciones del a�o.\\ Para ello contamos con tres entidades: ruta, clima y estaci�n. Estas tres siempres est�n vinculadas juntas. Es decir, para cada (ruta,clima) existe una instancia de estaci�n, por cada (clima, estaci�n) se vincula a una ruta y por cada par (ruta, estacion) hay un clima asociado. Por tal motivo es que decidimos modelarlo como una ternaria.
	\item Con lo que respecta al chofer y la licencia, nos pareci� m�s acorde separarlas en dos entidad pese a que en los requerimientos no queda clara dicha separaci�n. Lo realizamos as� para mayor claridad.
	\item Con respecto al dise�o de la entidad \textbf{Direcci�n} primero se opt� por hacerla d�bil de la ciudad.			
	
\begin{center}
\includegraphics[scale=0.50]{img/RecorridoDirDebil.png}
\end{center}

Al poner como clave primaria de \textbf{Direccion} su altura y nombre junto a la clave primaria de la ciudad, propag�bamos estas tres claves como for�neas en otras entidad vinculadas.

Luego optamos por dejar \textbf{Direcci�n} como una entidad fuerte con una clave primaria \textbf{codDir} para as� evitar la propagaci�n de claves y obtener un dise�o m�s sencillo.

\begin{center}
\includegraphics[scale=0.50]{img/RecorridoDirFuerte.png}
\end{center}

Vale aclarar que este dise�o admite tuplas duplicados con el mismo codDir, altura, nombre y c�digo de ciudad, sin embargo podemos impedir dicha situaci�n agregando una restricci�n adicional al DER.
  
\item Decidimos no guardar la historia de las reparaciones. Hubieramos podido hacerla de esta manera.

\begin{center}
\includegraphics[scale=0.50]{img/ReparacionConHistoria.png}
\end{center}

En este dise�o hubieramos necesitado agregar restricciones para asegurar que para toda fecha posterior a $Vehiculo.fechaAlta$ el $Vehiculo$ debe tener \textbf{solo y exactamente una} $Situacion$

Pero como no hay requerimientos expl�citos por los cuales necesitemos guardar la historia, decidimos simplificar el dise�o de la siguiente manera:

\begin{center}
\includegraphics[scale=0.50]{img/ReparacionSinHistoria.png}
\end{center}

\item Consideramos originalmente particionar los viajes en planificados y realizados, como se muestra en el siguiente diagrama:

\begin{center}
\includegraphics[scale=0.50]{img/viajePlanificado-Realizado.png}
\end{center}

Nos pareci� mejor considerar los viajes realizados como una especializaci�n de los planificados(todos los viajes son planificados y algunos de ellos adem�s pueden ser realizados). Esto lo dise�amos como una relaci�n de overlapping con una sola entidad llamada \textbf{Viaje Realizado}, describiendo que todos los viajes realizados heredan los mismos atributos que un viaje planificado, pero solo algunos viajes planificados tienen su correspondiente realizado.

\begin{center}
\includegraphics[scale=0.55]{img/viaje-Planificado-Realizado.png}
\end{center}

\item $(Chofer/Viaje Planificado)/Control$ es una agregaci�n pues no toda relacion $Chofer/Viaje Planificado$ tiene controles asociados.
Adem�s $Control$ es una Entidad asociada a la interrelacion entra $Vaje Planificado$ y $Control$ y no a una de ellas en particular.

\end{enumerate}

\subsection{DER}

\begin{center}
\includegraphics[height=0.95\textheight]{img/der.png}
\end{center}

\subsection{Restricciones}

\begin{enumerate}
	\item Dado 2 recorridos con distinto codigo, su origen y destino no pueden ser los mismos. Es decir, no puede haber dos recorridos ``iguales''con distinto c�digo.
	\item Para todo recorrido, el origen no puede ser igual al destino
	\item El recorrido de la ruta de todo viaje realizado debe ser igual al recorrido del viaje planificado.
	\item Para todo recorrido que tiene un viaje planificado asociado, existe una ruta.
	\item Para todo viaje la cantidad de conductores debe ser entre 1 y 3 y la fechaHoraLlegadaEst debe ser mayor a fechaHoraPartida.
	\item Para todo viaje realizado la fechaHoraLlegada debe ser mayor a fechaHoraPartida.
	\item Para todo viaje planificado que no esta realizado el vehiculo utilizado debe estar en uso.	
	\item Para todo vehiculo en reparacion la fecha de ingreso es mayor a la fecha de alta.
	\item Para toda licencia, la fecha de obtencion es menor a la fecha de vencimiento
	\item Para todo chofer, la fecha de nacimiento es menor que la fecha de obtenci�n de su licencia
	\item Para toda ruta, hay un �nico clima para todo per�odo definido
	\item Para todo control, su fecha de realizaci�n debe ser anterior a la fecha de partida del viaje que se est� controlando  
	\item La entidad Esaci�n puede tomar los valores, \textit{verano}, \textit{invierno}, \textit{oto�o} y \textit{primavera}
\end{enumerate}


\newpage

\section{MR}

\subsection{Esquema Relaci�n}

\begin{mr}{Agencia de Viajes}

\entidad{Licencia}{\pk{nroLicencia}, fechaObtencion, fechaVencimiento, observaciones}
\entidadPK{nroLicencia}
\entidadCK{nroLicencia}

\entidad{Chofer}{\pk{nroDocumento}, fechaNac, nomApe, domicilio, telefono, \fk{nroLicencia}}
\entidadPK{nroDocumento}
\entidadCK{nroDocumento}
\entidadFK{nroLicencia}

\entidad{Conduce}{\pfk{nroDocumento}, \pfk{codViaje}}
\entidadPK{(nroDocumento, codViaje)}
\entidadCK{(nroDocumento, codViaje)}
\entidadFK{nroDocumento, codViaje}

\entidad{Control}{\pk{codControl}, \fk{nroDocumento}, \fk{codViaje}, \fk{codTipo}, resultadoTest, fechaControl}
\entidadPK{codControl}
\entidadCK{codControl}
\entidadFK{(nroDocumento, codViaje), codTipo}

\entidad{TipoTest}{\pk{codTipo}, descripcion}
\entidadPK{codTipo}
\entidadCK{codTipo}

\entidad{Contingencia}{\pk{nroContingencia}, \pfk{codViaje}, descripcion}
\entidadPK{(nroContingencia, codViaje)}
\entidadCK{(nroContingencia, codViaje)}
\entidadFK{codViaje}

\entidad{Direccion}{\pk{codDir}, nombre, altura, \fk{codCiudad}}
\entidadPK{codDir}
\entidadCK{codDir}
\entidadFK{codCiudad}

\entidad{Ciudad}{\pk{codCiudad}, nombre}
\entidadPK{codCiudad}
\entidadCK{codCiudad}

\entidad{Recorrido}{\pk{codRecorrido}, nombre, \fk{codDirOrigien}, \fk{codDirDestino}}
\entidadPK{codRecorrido}
\entidadCK{codRecorrido}
\entidadFK{codDirOrigien, codDirDestino}

\entidad{Ruta}{\pk{nroRuta}, \pfk{codRecorrido}, cantKm, condicionesCamino, cantPeajes, tiempoEstimado}
\entidadPK{(nroRuta, codRecorrido)}
\entidadCK{(nroRuta, codRecorrido)}
\entidadFK{codRecorrido}

\entidad{Estado}{\pk{codEstado}, descripcion}
\entidadPK{codEstado}
\entidadCK{codEstado}

\entidad{Vehiculo}{\pk{nroPatente}, modelo, marca, capacidad, fechaAlta, \fk{codEstado}, enUso}
\entidadPK{nroPatente}
\entidadCK{nroPatente}
\entidadFK{codEstado}

\entidad{VehiculoEnReparacion}{\pfk{nroPatente}, fechaIngresoReparacion}
\entidadPK{nroPatente}
\entidadCK{nroPatente}
\entidadFK{nroPatente}

\entidad{ViajePlanificado}{\pk{codViaje}, fechaHoraPartida, fechaHoraLlegadaEst, \fk{nroPatente}, \fk{codRecorrido}}
\entidadPK{codViaje}
\entidadCK{codViaje}
\entidadFK{nroPatente,codRecorrido}

\entidad{ViajeRealizado}{\pk{codViaje}, fechaHoraLlegada, \fk{nroRuta}, \fk{codRutaRecorrido}}
\entidadPK{codViaje}
\entidadCK{codViaje}
\entidadFK{codViaje, (nroRuta, codRutaRecorrido)}

\entidad{Participa}{\pfk{codRecorrido}, \pfk{codRuta}, \pfk{nombreEstacion}, \fk{codClima}}
\entidadPK{((codRecorrido, codRuta), nombreEstacion)}
\entidadCK{((codRecorrido, codRuta), nombreEstacion)}
\entidadFK{(codRecorrido, codRuta), nombreEstacion, codClima}

\entidad{Estacion}{\pk{nombreEstacion}}
\entidadPK{nombreEstacion}
\entidadCK{nombreEstacion}

\entidad{Clima}{\pk{codClima}, descripcion}
\entidadPK{codClima}
\entidadCK{codClima}

\end{mr}

\subsection{Restricciones}

\paragraph{Nota: a menos que expl�citamente se mencione que un valor debe estar en otra entidad y viceversa, no vale la vuelta}

\begin{enumerate}
	\item $Chofer.nroLicencia$ debe estar en $Licencia.nroLicencia$
	\item $Conduce.codViaje$ debe estar en $ViajePlanificado.codViaje$ y viceversa
	\item $Conduce.nroDocumento$ debe estar en $Chofer.nroDocumento$
	\item $Control.nroDoumento$ debe estar en $Chofer.nroDocumento$
	\item $Control.codViaje$ debe estar en $ViajePlanificado.codViaje$
	\item $Control.codTipo$ debe estar en $TipoTest.codTipo$
	\item $Contingencia.codViaje$ debe estar en $ViajeRealizado.codViaje$
	\item $Direccion.codCiudad$ debe estar en $Ciudad.codCiudad$
	\item $Recorrido.codDirOrigen$ debe estar en $Direccion.codDir$
	\item $Recorrido.codDirDestino$ debe estar en $Direccion.codDir$
	\item $Ruta.codRecorrido$ debe estar en $Recorrido.codRecorrido$
	\item $VehiculoEnReparacion.codVehiculo$ debe estar en $Vehiculo.codVehiculo$
	\item $Vehiculo.codEstado$ debe estar en $Estado.codEstado$
	\item $Viaje.nroPatente$ debe estar en $Vehiculo.nroPatente$
	\item $Viaje.codRecorrido$ debe estar en $Recorrido.codRecorrido$
	\item $ViajeRealizado.codViaje$ debe estar en $ViajePlanificado.codViaje$
	\item $ViajeRealizado.nroRuta$ debe estar en $Ruta.nroRuta$
	\item $ViajeRealizado.codRutaRecorrido$ debe estar en $Recorrido.codRecorrido$
	\item $Participa.codRecorrido$ debe estar en $Recorrido.codRecorrido$
	\item $Participa.nombreEstacion$ debe estar en $Estacion.nombreEstacion$
	\item $Participa.codClima$ debe estar en $Clima.codClima$
	\item $Licencia.fechaObtencion$ es anterior a $Licencia.fechaVencimiento$
	\item Si $Vehiculo.enUso$ es false, entonces debe existir $VehiculoEnReparacion$
	\item $Vehiculo.enUso$ toma valores booleanos ( $1=enUso$ o $0=noEnUso$ )
	\item $\forall r1, r1 \in Recorrido (\not\exists r2, r2 \in Recorrido (r1.codRecorrido \neq r2.codRecorrido$  $\wedge$ \\
	$r2.codDirDestino = r1.codDirDestino$ $\wedge$   \\
	$r2.codDirOrigen = r1.codDirOrigen))$ \footnote{Se corresponde con restricci�n \ref{MER1} del MER} 
	\item $Recorrido.codDirOrigen \neq Recorrido.codDirDestino$	
	\footnote{Se corresponde con restricci�n \ref{MER2} del MER}
	\item $\forall v1, v1 \in ViajeRealizado(\exists rut1, rut1 \in Ruta(rut1.codRuta = v1.codRuta \wedge v1.codRecorrido = rut1.codRecorrido))$ 
	\footnote{Se corresponde con restricci�n \ref{MER3} del MER}
	\item $\forall rec, vp( rec \in Recorrido \wedge vp \in ViajePlanificado \wedge vp.codRecorrido = rec.codRecorrido \Rightarrow \exists rut,rut \in Ruta \wedge rut.codRecorrido = rec.codRecorrido)$
\footnote{Se corresponde con restricci�n \ref{MER4} del MER}	
	% vincular los dos siguientes a la restrccion 5 del MER
	\item 	$\not\exists c1,c2,c3,c4 (c1,c2,c3,c4 \in Conduce \wedge 	c1.codViaje = c2.codViaje = c3.codViaje = c4.codViaje )$
	\footnote{Se corresponde con restricci�n \ref{MER5} del MER}
	\item 	$\forall v ( v \in ViajePlanificado \wedge v.fechaHoraLlegadaEst > v.fechaHoraPartida )$
	\footnote{Se corresponde con restricci�n \ref{MER5} del MER}
	\item $\forall vr \in ViajeRealizado ( \exists vp \in ViajePlanificado (vr.codViaje = vp.codViaje \wedge vr.fechaHoraLlegada > vp.fechaHoraPartida))$
	\footnote{Se corresponde con restricci�n \ref{MER6} del MER}
	\item $(\forall vp \in ViajePlanificado \wedge $$\not\exists$$ vr \in ViajeRealizado vr.codViaje=vp.codViaje) \Rightarrow \exists veh \in Vehiculo \wedge veh.nroPatente = vp.nroPatente \wedge veh.enUso = true$ 
	\footnote{Se corresponde con restricci�n \ref{MER7} del MER}
	\item $\forall veh \in VehiculoEnReparacion \wedge veh.fechaIngresoReparacion > veh.fechaAlta$
\footnote{Se corresponde con restricci�n \ref{MER8} del MER}
	\item $\forall lic \in Licencia (lic.fechaObtencion < lic.fechaVencimiento)$
	\footnote{Se corresponde con restricci�n \ref{MER9} del MER}
	\item $\forall cho,lic ( cho \in Chofer \wedge lic \in Licencia \wedge cho.nroLicencia = lic.nroLicencia \wedge cho.fechaNac < lic.fechaObtencion)$
	\footnote{Se corresponde con restricci�n \ref{MER10} del MER}
	\item $\forall r1,est , r1 \in Ruta \wedge est \in Estacion (\exists p \in Participa \wedge p.nroRuta = r1.nroRuta \wedge p.codRecorrido = r1.codRecorrido \wedge est.nombreEstacion = p.nombreEstacion )$	
	\footnote{Se corresponde con restricci�n \ref{MER11} del MER}
	\item $\forall c \in Control \exists vp,cond  vp \in ViajePlanificado  \wedge cond \in Conduce \wedge cond.codViaje = vp.codViaje \wedge cond.codControl = c.codControl \wedge vp.fechaHoraPartida  < c.fechaControl$
\footnote{Se corresponde con restricci�n \ref{MER12} del MER}
\item $\forall est \in Estacion (est.nombreEstacion ='Verano' \vee est.nombreEstacion ='Invierno' \vee est.nombreEstacion ='Otonio' \vee est.nombreEstacion ='Primavera')$
\footnote{Se corresponde con restricci�n \ref{MER13} del MER}
\end{enumerate}


\newpage

\section{Dise�o F�sico}
\subsection{Aclaraciones}
\begin{itemize}
	\item Vehiculo en reparaci�n no se pone con un esquema relaci�n sino que se pone la restricci�n del tipo de vehiculo en funci�n de la fecha de ingreso a reparaci�n. 
	\item En la tabla \textbf{control} el campo \textit{resultadoTest} es de tipo bit ya que los test pueden estar aprobados o desaprobador. Es decir toma los valores $0$ y $1$, donde $1$ significa aprobado y $0$ significa desaprobado.
	\item Los vehiculos pueden estar \textbf{en uso} o \textbf{en reparaci�n}. En la tabla \textbf{vehiculo} se tiene un campo llamado \textit{EnUso}. Este campo es de tipo bit ya que toma solo dos valores: $1$ significa en uso y $0$ significa en reparaci�n.
	\item En todos los campos de tipo char, estos se tomaron de longitud 30.   
\end{itemize}

\subsection{Implementaci�n}

Para el presente trabajo se decidi� utilizar el motor de base de datos SQL Server 2005. El mismo se encuentra disponible en las computadores de los laboratorios del departamento de computaci�n por lo que en los mismos se podr� regenerar los scripts de creaci�n, inserci�n y consultas entregados en el cd adjunto a este documento.

\subsubsection{Consultas}
// tipear con negrita las consultas y decir como se resolvieron

\begin{enumerate}
\item \textbf{Los recorridos para los cuales se usaron todas las rutas posibles registradas para ese recorrido, para viajes realizados el a�o pasado y recorridos asociados a m�s de una ruta}
\item \textbf{El promedio de los viajes realizados por veh�culo por a�o y el estado en que �ste se encuentra}
\item \textbf{Los choferes que han utilizado todos los veh�culos de menos de dos a�os de antig�edad, en viajes del mismo semestre. }
\end{enumerate}



\subsubsection{Triggers}
// decidir que triggers se implementaron y c�mo se hicieron 

\begin{itemize}
\item \FALTA
\end{itemize}


\end{document}




