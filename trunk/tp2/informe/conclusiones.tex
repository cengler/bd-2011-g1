\section{Conclusiones}

\subsection{\fs}
Vemos que ninguna de las estrategias de reemplazo de p�ginas es mejor a otra.
\newline
Pero uno puede observar resultados interesantes cuando se necesita recorrer m�s de una vez la traza, dado que ahora hay oportunidad de cachear p�ginas pedidas anteriormente.
\newline
Para el caso de dos lecturas sucesivas sobre la misma traza podemos considerar dos escenarios:
\begin{itemize}
 \item    \textbf{La cantidad de frames en buffer es menor a la cantidad de p�ginas pedidas}: la �nica estrategia que tiene Hit-Rate mayor a 0 es MRU. FIFO y LRU cuando necesitan alojar una nueva p�gina empiezan desalojando la primera en ser almacenada y este comportamiento impide reaprovechar esas p�ginas en la segunda lectura de la traza.
 \item    \textbf{La cantidad de frames en buffer en mayor o igual a la cantidad de p�ginas pedidas}: en este escenario las tres estrategias muestran un HitRate del 50\% dado que si bien en la primera lectura no tienen ninguna p�gina en buffer, para la segunda no tuvieron que desalojar ninguna (son todos hits).
\end{itemize}

\subsection{\isc}
Si suponemos que hay un �ndice clustered sobre un atributo $a$ de una relaci�n $A$ y la solicitud que se hizo es para un rango $c<a<b$. En este caso, el HitRate es igual a 0 dado que accedo por el �ndice al bloque donde se encuentra $c$ y recorro todos los bloques hasta que encuentro $d$, por lo tanto no hay posibilidad de tener cacheado bloques previamente solicitados.\newline
Hay casos particulares donde el uso repetido de \isc implica un HitRate mayor a 0. Ejemplos de estos casos son:
\begin{itemize}
	\item Dos lecturas sucesivas sobre el mismo rango. El HitRate de esta operaci�n dar�a resultados an�logos a los obtenidos con \fs cuando hicimos dos lecturas seguidas sobre el mismo archivo.
	\item Casos donde se realizan lecturas de rangos con intersecci�n distinta a vac�o, lo cual puede permitir la reutilizaci�n de bloques pertenecientes a la intersecci�n de estos rangos si los mismos se encuentran en el Buffer.
\end{itemize}

\subsection{\isu}
  En este algoritmo no se puede predecir qu� estrategia de reemplazo de frames es conveniente, pero a diferencia del \isc se puede observar que es posible que se resoliciten bloques ya cacheados anteriormente. Es decir,la coincidencia de que una p�gina ya est� en memoria depende de la traza y de la estrategia de reeamplazo que se est� utilizando. \newline

\subsection{\bnlj}
Podemos considerar los siguientes escenarios:
\begin{itemize}
  \item \textbf{Las p�ginas de R pueden ser cacheadas en memoria en su totalidad} dejando por lo menos dos frames de memoria libres, uno para S y otro para el resultado del JOIN. Este caso no es muy interesante, ya que con cualquier algoritmo de reemplazo el Hit-Rate es cero. La explicaci�n es la siguiente: tenemos cacheado R completamente y lo que necesitamos ir iterando es el bloque de S contra el cu�l cruzamos los bloques de R. \newline
  Es decir:
  \begin{itemize}
    \item En la primera iteraci�n cruzamos todos los bloques de R contra el primero de S y obtenemos un Miss (ninguno estaba cargado previamente)
    \item Cargamos en memoria el segundo bloque de S
    \item De ser necesario reemplazar una p�gina, hay que tener en cuenta que el �nico frame que hizo release fue el anterior de S y volvemos a obtener otro Miss correspondiente al nuevo bloque de S reci�n cargado
    \item Terminamos de iterar sobre S sin haber reutilizado ning�n frame (todos Misses).
  \end{itemize}
  \item \textbf{La totalidad de las p�ginas de R no pueden ser cacheadas en memoria} y se cuenta con un buffer de tama�o fijo.\\
La idea de las pruebas realizadas fue variar los bloques asignados a R de tal forma que sobren bloques de memoria y de esta manera ver que ocurre cuando se tiene m�s de un bloque libre de memoria para cachear bloques de S. Esto no sucede en el algoritmo tradicional de \bnlj ya que en �ste solo se le dedica un bloque de memoria a S. \newline
Como resultado de las pruebas obtuvimos que, en esta variaci�n de \bnlj, se pueden llegar a tener cacheados, dependiendo de las estrategias de reemplazo, algunos bloques de S que son solicitados nuevamente y obtener as� un hit rate mayor a cero. \newline
Se puede concluir que MRU es la estrategia de reemplazo que mejor trabaja en \bnlj, tal como se ve en las Figuras 2 y 3 de la secci�n 4.4.





\end{itemize}

