
\section{Implementaci�n de estrategias}

\subsection{Implementaci�n de LRU y MRU}

En el paquete \textbf{ubadb.components.bufferManager.bufferPool.replacementStrategies.lru} creamos las clases: 
\begin{itemize}
	\item LRUBufferFrame \\ Esta clase extiende BufferFrame para manener para cada frame un nroAcceso secuencial, para luego poder en el strategy decidir el mas antiguo.
	\item LRUReplacementStrategy \\ Implementaci�n del algoritmo LRU
\end{itemize}

El el paquete \textbf{ubadb.components.bufferManager.bufferPool.replacementStrategies.lru} del directorio src/test creamos los siguientes test de unidad: 
\begin{itemize}
	\item LRUBufferFrameTest \\ Verifica dado una secuencia de creaciones, pin y unpin, los contadores queden setados correctamente.
	\item LRUReplacementStrategyTest \\ Verifica que los frames seleccionados por el algoritmo se den segun lo esperado por distintas trazas determinadas.
\end{itemize}

Todo lo comenzado se hace tambi�n para la politica de reemplazo MRU en el paquete \\ \textbf{ubadb.components.bufferManager.bufferPool.replacementStrategies.mru}

\subsection{Implementaci�n extras}

\begin{itemize}
	\item Creamos la clase \textbf{ubadb.apps.bufferManagement.PageReferenceTraceReader} para poder parsear trazas como las dadas de ejemplo por la c�tedra.
	\item Creamos la clase \textbf{ubadb.apps.bufferManagement.DemoStrategy} para ejecutar todas las trazas de un directorio variando la politica de reemplazo de paginas y el tama�o del SingleBuffer.
	\item Sobreescribimos el m�todo \verb|toString()| de \textsl{SingleBufferPool} para poder imprimirlo por pantalla y estudiar la disposici�n del buffer.
\end{itemize}
