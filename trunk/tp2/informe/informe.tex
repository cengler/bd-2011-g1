\documentclass[a4paper, 10pt, notitlepage]{article}

\usepackage{moreverb} %para importar codigo

\usepackage{pepotina} %paquete personal para la caratula del DC

\usepackage[spanish,activeacute]{babel}
\usepackage{babel} %paquete de idioma

\usepackage[latin1]{inputenc}

\usepackage[normalem]{ulem}
\usepackage{ulem}

%\usepackage{color}

\usepackage{hyperref}
%\usepackage[all]{hypcap}

\usepackage{caeycaeBD}

\usepackage{fancyhdr} %linea sup con comentarios

\usepackage{lscape} %para hoja apaisada

\usepackage{framed} %para crear cajas de texto

\usepackage{lastpage} %ultima pagina

%\usepackage{pstricks}
%\usepackage{uml} %UML

\usepackage{listings}
%\lstset{
%  breaklines=true,                                     % line wrapping on
%  language=ocl,
%  frame=ltrb,
%  framesep=5pt,
%  basicstyle=\normalsize,
%  keywordstyle=\ttfamily\color{OliveGreen},
%  identifierstyle=\ttfamily\color{CadetBlue}\bfseries,
%  commentstyle=\color{Brown},
%  stringstyle=\ttfamily,
%  showstringspaces=ture
%}

\addtolength{\topmargin}{-50pt} 
\addtolength{\textwidth}{160pt}
\addtolength{\textheight}{120pt}
\addtolength{\oddsidemargin}{-70pt}

%\newcommand{\minix}{\textsl{minix }}

%%% Encabezado y pie de p'agina
\pagestyle{fancy}
\fancyhead[LO]{Base de Datos}
\fancyhead[C]{}
\fancyhead[RO]{P\'agina \thepage\ de \pageref{LastPage}}
\renewcommand{\headrulewidth}{0.4pt}
\fancyfoot{}

\newcommand{\FALTA}{{\bf FALTA FALTA FALTA FALTA FALTA FALTA FALTA FALTA }}

\newcommand{\fs}{\textsl{File Scan }}
\newcommand{\isc}{\textsl{Index Scan Clustered }}
\newcommand{\isu}{\textsl{Index Scan Unclustered }}
\newcommand{\bnlj}{\textsl{Block Nested Loops Join }}


\def\var#1{\textsl{#1}}


\begin{document}

\universidad{Universidad de Buenos Aires}
\facultad{Facultad de Ciencias Exactas y Naturales}
\departamento{Departamento de Computaci�n}
\materia{Base de Datos}
\resumen{El presente trabajo eval�a el impacto de la utilizaci�n de diferentes estrategias de reemplazo de p�ginas en la implementaci�n de ciertas operaciones sobre una BD. Se analiza la importancia del m�dulo Buffer Manager en el contexto de un motor de  BD.}
\keys{Base de Datos, Base de Datos Relacional, Buffer Manager, LRU, MRU, FIFO}
\titulo{Tp2}
\subtitulo{Buffer Manager}
\grupo{N�mero de Grupo: 1}
\fecha{2do Cuatrimestre 2011}
\footspace{1cm}
\integrante{Piotrkowski, Kevin}{204/06}{marcoskevin@gmail.com}
\integrante{Rugnone, Mart�n}{665/04}{mrugnone@hotmail.com}
\integrante{Alvarez, Mar�a de los Angeles}{264/05}{mdelosaalvarez@hotmail.com}
\integrante{Engler, Christian}{314/05}{caeycae@gmail.com}

\maketitle{}

\tableofcontents

\newpage


\section{Introducci�n}

\begin{quotation}
\textit{``El Buffer Manager es uno de los componentes m�s importantes dentro de un motor de BD. Su principal funci�n es administrar un espacio de memoria de la BD, utilizado como una especie de memoria cach�. El objetivo es que las diferentes aplicaciones que usan la BD y requieren p�ginas de disco, puedan recuperar la p�gina de este espacio de memoria y accedan lo menos posible al disco.''}
\end{quotation}

En este informe analizaremos el tipo de trazas definidas por distintos algoritmos(NBLJ, File Scan, Index Scan, etc) y las diferencias de permonance(medida seg�n el hit-rate) que se obtienen al probar las estrategias de reemplazo MRU, LRU y FIFO. Los resultados obtenidos por las diferentes estrategias de reemplazo nos permitir�n sacar conclusiones y definir heur�sticas sobre la mejor estrategia a usar para cada algoritmo.

En la realizaci�n de este trabajo pr�ctico empezamos por realizar la implementaci�n de las estrategias LRU y MRU con sus respectivos casos de test, desarrollamos un parser de trazas almacenadas en archivos de texto plano, definimos trazas representativas de los algorimos a evaluar, luego analizamos las trazas definidas con las 3 estrategias de reemplazo mencionadas anteriormente, para finalmente expresar las conclusiones obtenidas durante la realizaci�n del trabajo pr�ctico.

\section{Hip�tesis}
Consideramos los RELEASE de los bloques como una referencia al mismo por lo cu�l inciden en las distintas estrategias de reemplazo.
Ejemplo de traza:
REQUEST [R,1]
REQUEST [R,2]
RELEASE [R,2]
RELEASE [R,1]

Si se utilizara una estrategia de reemplazo MRU, el primero en ser desalojado ser�a el bloque 1 de R. 
En cambio, si la estrategia de reemplazo es LRU, el bloque que se desaloja es el 2.




\newpage


\section{Implementaci�n de estrategias}

\subsection{Implementaci�n de LRU y MRU}

En el paquete \textbf{ubadb.components.bufferManager.bufferPool.replacementStrategies.lru} creamos las \\clases: 
\begin{itemize}
	\item LRUBufferFrame \\ Esta clase extiende BufferFrame para mantener para cada frame un nroAcceso secuencial, para luego poder en el strategy decidir el m�s antiguo.
	\item LRUReplacementStrategy \\ Implementaci�n del algoritmo LRU
\end{itemize}

En el paquete \textbf{ubadb.components.bufferManager.bufferPool.replacementStrategies.lru} del directorio src/test creamos los siguientes test de unidad: 
\begin{itemize}
	\item LRUBufferFrameTest \\ Verifica dada una secuencia de creaciones, pins y unpins, que los contadores queden seteados correctamente.
	\item LRUReplacementStrategyTest \\ Verifica que los frames seleccionados por el algoritmo se den seg�n lo esperado por distintas trazas determinadas.
\end{itemize}

Todo lo comentado se hace tambi�n para la pol�tica de reemplazo MRU en el paquete \\ \textbf{ubadb.components.bufferManager.bufferPool.replacementStrategies.mru}

\subsection{Implementaci�n extras}

\begin{itemize}
	\item Creamos la clase \textbf{ubadb.apps.bufferManagement.PageReferenceTraceReader} para poder parsear trazas como las dadas de ejemplo por la c�tedra.
	\item Creamos la clase \textbf{ubadb.apps.bufferManagement.DemoStrategy} para ejecutar todas las trazas de un directorio variando la pol�tica de reemplazo de p�ginas y el tama�o del SingleBuffer.
	\item Sobrescribimos el m�todo \verb|toString()| de \textsl{SingleBufferPool} para poder imprimirlo por pantalla y estudiar la disposici�n del buffer.
\end{itemize}

\subsection{Modificaciones en SingleBufferPool}
\begin{itemize}
	\item Implementamos el m�todo toString() para imprimir el estado del buffer con las p�ginas actualmente almacenadas en el mismo.
	\item Asociamos un orden a cada frame almacenado para poder imprimir el buffer con los frames seg�n el orden en que fueron siendo agregados.
\end{itemize}

\subsection{Ejecuci�n del c�digo}
Pasos para la correcta ejecuci�n del programa desde eclipse:
\begin{itemize}
	\item Configurar en Run $>$ Run Configurations $>$ Arguments $>$ Working Directory: la carpeta TARGET correspondiente al c�digo.
	\item Ir a la clase ubadb $>$  apps $>$  bufferManagement $>$ DemoStrategy y correr el proyecto desde esta clase.
\end{itemize}	


Pasos para la correcta ejecuci�n del programa desde consola. Ejecutar los siguientes comandos:
\begin{itemize}
	\item 
			\begin{verbatim}
				mvn clean
			\end{verbatim}
	\item 
			\begin{verbatim}
				mvn install
			\end{verbatim}
		\item 
			\begin{verbatim}
				cd target
			\end{verbatim}
		\item 
			\begin{verbatim}
				java -jar ubadb-1.0.jar
			\end{verbatim}
\end{itemize}	


\newpage


\section{Evaluaci�n de estrategias}

\subsection{File Scan}

La traza del \fs no tendr� HITs para ning�n tipo de algoritmo de reemplazo de p�ginas porque \fs pide UNA sola vez cada p�gina.\\
Por otro lado, se observaron resultados interesantes cuando hac�amos traces que le�an 2 veces el mismo archivo y comparab�mos los hit-rates de las diferentes estatregias, de estos resultados obtuvimos las siguientes conclusiones:

\begin{itemize}
\item LRU y FIFO tienen el mismo comportamiento durante un \fs porque l�s p�ginas que se eligen para desalojar en ambos casos son las primeras en haber sido referidas.
\item Si hab�a igual o m�s cantidad de frames, el hit-rate pasaba a ser del 50\% para todas las estrategias, porque si bien en la primera pasada todas las referencias a p�ginas era misses, para la segunda estaban todos en memoria.
\item Para una cantidad de p�ginas superior en una unidad a la cantidad de frames, LRU ten�a 100\% de miss-rate(y por ende tambi�n la estrategia FIFO). Esto se debe a que necesita desalojar el primer frame para alojar la �ltima p�gina(en la primera pasada); luego, cuando necesitaba leer nuevamente la primera p�gina(en la segunda pasada) tiene que desalojar el �ltimo frame en ser accedido(el segundo), para el segundo el tercero, y as� sucesivamente.
\item Las estrategias empiezan a convergen en torno a un hit-rate del 50\% cuando la cantidad de frames se acerca a la cantidad de p�ginas dado que es necesario desalojar menos frames.
\item Entre las estrategias evaluadas, la �nica que mejora el hit-rate cuando la cantidad de frames es menor a la cantidad de p�ginas a ser le�das es MRU. 
\end{itemize}


\subsection{Index Scan Clustered}





\newpage

\section{Conclusiones}

\subsection{\fs}
Vemos que ninguna de las estrategias de reemplazo de p�ginas es mejor a otra.
\newline
Pero uno puede observar resultados interesantes cuando se necesita recorrer m�s de una vez la traza, dado que ahora hay oportunidad de cachear p �ginas pedidas anteriormente.
\newline
Para el caso de dos lecturas sucesivas sobre la misma traza podemos considerar dos escenarios:
\begin{itemize}
 \item    \textbf{La cantidad de frames en buffer es menor a la cantidad de p�ginas pedidas}: la �nica estrategia que tiene Hit-Rate mayor a 0 es MRU. FIFO y LRU cuando necesitan alojar una nueva p�gina empiezan desalojando la primera en ser almacenada y este comportamiento impide reaprovechar esas p�ginas en la segunda lectura de la traza.
 \item    \textbf{La cantidad de frames en buffer en mayor o igual a la cantidad de p�ginas pedidas}: en este escenario las tres estrategias muestran un HitRate del 50\% dado que si bien en la primera lectura no tienen ninguna p�gina en buffer, para la segunda no tuvieron que desalojar ninguna(son todos hits).
\end{itemize}

\subsection{\isc}
Si suponemos que hay un �ndice clustered sobre un atributo $a$ de una relaci�n $A$ y la solicitud que se hizo es para un rango $c<a<b$. En este caso, el HitRate es igual a 0 dado que accedo por el �ndice al bloque donde se encuentra $c$ y recorro todos los bloques hasta que encuentro $d$, por lo tanto no hay posibilidad de tener cacheado bloques previamente solicitados.\newline
Hay casos particulares donde el uso repetido de \isc implica un HitRate mayor a 0. Ejemplos de estos casos son:
\begin{itemize}
	\item Dos lecturas sucesivas sobre el mismo rango. El HitRate de esta operaci�n dar�a resultados an�logos a los obtenidos con \fs cuando hicimos dos lecturas seguidas sobre el mismo archivo.
	\item Casos donde se realizan lecturas de rangos con intersecci�n distinta a vac�o, lo cual puede permitir la reutilizaci�n de bloques pertenecientes a la intersecci�n de estos rangos si los mismos se encuentran en el Buffer.
\end{itemize}

\subsection{\isu}
  En este algoritmo no se puede predecir qu� estrategia de reemplazo de frames es conveniente, pero a diferencia del \isc se puede observar que es posible que se resoliciten bloques ya cacheados anteriormente. Es decir,la coincidencia de que una p�gina ya est� en memoria depende de la traza y de la estrategia de reeamplazo que se est� utilizando. \newline

\subsection{\bnlj}
Podemos considerar los siguientes escenarios:
\begin{itemize}
  \item \textbf{Las p�ginas de R pueden ser cacheadas en memoria en su totalidad} dejando por lo menos dos frames de memoria libres, uno para S y otro para el resultado del JOIN. Este caso no es muy interesante, ya que con cualquier algoritmo de reemplazo el Hit-Rate es cero. La explicaci�n es la siguiente: tenemos cacheado R completamente y lo que necesitamos ir iterando es el bloque de S contra el cu�l cruzamos los bloques de R. \newline
  Es decir:
  \begin{itemize}
    \item En la primera iteraci�n cruzamos todos los bloques de R contra el primero de S y obtenemos un Miss(ninguno estaba cargado previamente)
    \item Cargamos en memoria el segundo bloque de S
    \item De ser necesario reemplazar una p�gina, hay que tener en cuenta que el �nico frame que hizo release fue el anterior de S y volvemos a obtener otro Miss correspondiente al nuevo bloque de S reci�n cargado
    \item Terminamos de iterar sobre S sin haber reutilizado ning�n frame(todos Misses).
  \end{itemize}
  \item \textbf{La totalidad de las p�ginas de R no pueden ser cacheadas en memoria} y se cuenta con un buffer de tama�o fijo.\\
La idea de las pruebas realizadas fue variar los bloques asignados a R de tal forma que sobren bloques de memoria y de esta manera ver que ocurre cuando se tiene m�s de un bloque libre de memoria para cachear bloques de S. Esto no sucede en el algoritmo tradicional de \bnlj ya que en �ste solo se le dedica un bloque de memoria a S. \newline
Como resultado de las pruebas obtuvimos que, en esta variaci�n de \bnlj, se pueden llegar a tener cacheados, dependiendo de las estrategias de reemplazo, algunos bloques de S que son solicitados nuevamente y obtener as� un hit rate mayor a cero. \newline
Se puede concluir que MRU es la estrategia de reemplazo que mejor trabaja en \bnlj, tal como se ve en las Figuras 2 y 3 de la secci�n 4.4.





\end{itemize}



\end{document}

