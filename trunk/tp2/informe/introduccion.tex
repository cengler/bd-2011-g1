
\section{Introducci�n}

\begin{quotation}
\textit{``El Buffer Manager es uno de los componentes m�s importantes dentro de un motor de BD. Su principal funci�n es administrar un espacio de memoria de la BD, utilizado como una especie de memoria cach�. El objetivo es que las diferentes aplicaciones que usan la BD y requieren p�ginas de disco, puedan recuperar la p�gina de este espacio de memoria y accedan lo menos posible al disco.''}
\end{quotation}

En este informe analizaremos el tipo de trazas definidas por distintos algoritmos(NBLJ, File Scan, Index Scan, etc) y las diferencias de permonance(medida seg�n el hit-rate) que se obtienen al probar las estrategias de reemplazo MRU, LRU y FIFO. Los resultados obtenidos por las diferentes estrategias de reemplazo nos permitir�n sacar conclusiones y definir heur�sticas sobre la mejor estrategia a usar para cada algoritmo.

En el desarrollo de este trabajo pr�ctico empezamos por realizar la implementaci�n de las estrategias LRU y MRU con sus respectivos casos de test, desarrollamos un parser de trazas almacenadas en archivos de texto plano, definimos trazas representativas de los algorimos a evaluar, luego analizamos las trazas definidas con las 3 estrategias de reemplazo mencionadas anteriormente, para finalmente expresar las conclusiones obtenidas durante la realizaci�n del trabajo pr�ctico.

\section{Hip�tesis}
\begin{itemize}
 \item Consideramos los RELEASE de los bloques como una referencia al mismo por lo cu�l inciden en las distintas estrategias de reemplazo.
Ejemplo de traza:
REQUEST [R,1]
REQUEST [R,2]
RELEASE [R,2]
RELEASE [R,1]

Si se utilizara una estrategia de reemplazo MRU, el primero en ser desalojado ser�a el bloque 1 de R. 
En cambio, si la estrategia de reemplazo es LRU, el bloque que se desaloja es el 2.

\item Al momento de evaluar juntas usando algoritmo BNLJ asumimos que contamos con un bloque extra para guardar el resultado de la misma. Esto se debe a que el BufferManager reserva bloques de memoria solo cuando una transaccion hace un request de una p�gina en disco.
Ejemplo: si queremos hacer un join entre dos relaciones A y C y contamos con 100 bloques en memoria, el algoritmo indica que tenemos que dedicar 1 bloque para A y (100-2) para C dejando uno libre para el resultado. Como asumimos que tenemos un bloque extra de memoria para el resultado necesitar�amos dedicar 1 bloque para A y (101-2) para C.
\end{itemize}
