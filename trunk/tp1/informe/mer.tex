\section{MER}

\subsection{Decisiones de Dise�o}

\begin{enumerate}
	\item Ruta es d�bil en relaci�n a un recorrido porque s�lo sirve como una forma de ir desde el origen del recorrido a su destino.
	\item Contingencia es una entidad d�bil porque puede haber una cantidad arbitraria de ellas por cada viaje realizado(en contraposici�n a una lista de contingencias predefinidas ej:choque, bloqueo de ruta, control policial, etc) y las mismas no tienen sentido sin su viaje relacionado.
	\item La ruta tiene asociado un clima por cada periodo del a�o. Como se menciono en las hipotesis el periodo son las estaciones del a�o.\\ Para ello contamos con tres entidades: ruta, clima y estaci�n. Estas tres siempres est�n vinculadas juntas. Es decir, para cada (ruta,clima) existe una instancia de estaci�n, por cada (clima, estaci�n) se vincula a una ruta y por cada par (ruta, estacion) hay un clima asociado. Por tal motivo es que decidimos modelarlo como una ternaria.
	\item Con lo que respecta al chofer y la licencia, nos pareci� m�s acorde separarlas en dos entidad pese a que en los requerimientos no queda clara dicha separaci�n. Lo realizamos as� para mayor claridad.
	\item Con respecto al dise�o de la entidad \textbf{Direcci�n} primero se opt� por hacerla d�bil de la ciudad.			
	
\begin{center}
\includegraphics[scale=0.50]{img/RecorridoDirDebil.png}
\end{center}

Al poner como clave primaria de \textbf{Direccion} su altura y nombre junto a la clave primaria de la ciudad, propag�bamos estas tres claves como for�neas en otras entidad vinculadas.

Luego optamos por dejar \textbf{Direcci�n} como una entidad fuerte con una clave primaria \textbf{codDir} para as� evitar la propagaci�n de claves y obtener un dise�o m�s sencillo.

\begin{center}
\includegraphics[scale=0.50]{img/RecorridoDirFuerte.png}
\end{center}

Vale aclarar que este dise�o admite tuplas duplicados con el mismo codDir, altura, nombre y c�digo de ciudad, sin embargo podemos impedir dicha situaci�n agregando una restricci�n adicional al DER.
  
\item Decidimos no guardar la historia de las reparaciones. Hubieramos podido hacerla de esta manera.

\begin{center}
\includegraphics[scale=0.50]{img/ReparacionConHistoria.png}
\end{center}

En este dise�o hubieramos necesitado agregar restricciones para asegurar que para toda fecha posterior a $Vehiculo.fechaAlta$ el $Vehiculo$ debe tener \textbf{solo y exactamente una} $Situacion$

Pero como no hay requerimientos expl�citos por los cuales necesitemos guardar la historia, decidimos simplificar el dise�o de la siguiente manera:

\begin{center}
\includegraphics[scale=0.50]{img/ReparacionSinHistoria.png}
\end{center}

\item Consideramos originalmente particionar los viajes en planificados y realizados, como se muestra en el siguiente diagrama:

\begin{center}
\includegraphics[scale=0.50]{img/viajePlanificado-Realizado.png}
\end{center}

Nos pareci� mejor considerar los viajes realizados como una especializaci�n de los planificados(todos los viajes son planificados y algunos de ellos adem�s pueden ser realizados). Esto lo dise�amos como una relaci�n de overlapping con una sola entidad llamada \textbf{Viaje Realizado}, describiendo que todos los viajes realizados heredan los mismos atributos que un viaje planificado, pero solo algunos viajes planificados tienen su correspondiente realizado.

\begin{center}
\includegraphics[scale=0.55]{img/viaje-Planificado-Realizado.png}
\end{center}

\item $(Chofer/Viaje Planificado)/Control$ es una agregaci�n pues no toda relacion $Chofer/Viaje Planificado$ tiene controles asociados.
Adem�s $Control$ es una Entidad asociada a la interrelacion entra $Vaje Planificado$ y $Control$ y no a una de ellas en particular.

\end{enumerate}

\subsection{DER}

\begin{center}
\includegraphics[height=0.95\textheight]{img/der.png}
\end{center}

\subsection{Restricciones}

\begin{enumerate}
	\item Dado 2 recorridos con distinto codigo, su origen y destino no pueden ser los mismos. Es decir, no puede haber dos recorridos ``iguales''con distinto c�digo.
	\item Para todo recorrido, el origen no puede ser igual al destino
	\item El recorrido de la ruta de todo viaje realizado debe ser igual al recorrido del viaje planificado.
	\item Para todo recorrido que tiene un viaje planificado asociado, existe una ruta.
	\item Para todo viaje la cantidad de conductores debe ser entre 1 y 3 y la fechaHoraLlegadaEst debe ser mayor a fechaHoraPartida.
	\item Para todo viaje realizado la fechaHoraLlegada debe ser mayor a fechaHoraPartida.
	\item Para todo viaje planificado que no esta realizado el vehiculo utilizado debe estar en uso.	
	\item Para todo vehiculo en reparacion la fecha de ingreso es mayor a la fecha de alta.
	\item Para toda licencia, la fecha de obtencion es menor a la fecha de vencimiento
	\item Para todo chofer, la fecha de nacimiento es menor que la fecha de obtenci�n de su licencia
	\item Para toda ruta, hay un �nico clima para todo per�odo definido
	\item Para todo control, su fecha de realizaci�n debe ser anterior a la fecha de partida del viaje que se est� controlando  
	\item La entidad Esaci�n puede tomar los valores, \textit{verano}, \textit{invierno}, \textit{oto�o} y \textit{primavera}
\end{enumerate}
