\documentclass[a4paper, 10pt, notitlepage]{article}

\usepackage{moreverb} %para importar codigo

\usepackage{pepotina} %paquete personal para la caratula del DC

\usepackage[spanish,activeacute]{babel}
\usepackage{babel} %paquete de idioma

\usepackage[latin1]{inputenc}

\usepackage[normalem]{ulem}
\usepackage{ulem}

%\usepackage{color}

\usepackage{hyperref}
%\usepackage[all]{hypcap}

\usepackage{caeycaeBD}

\usepackage{fancyhdr} %linea sup con comentarios

\usepackage{lscape} %para hoja apaisada

\usepackage{framed} %para crear cajas de texto

\usepackage{lastpage} %ultima pagina

%\usepackage{pstricks}
%\usepackage{uml} %UML

\usepackage{listings}
%\lstset{
%  breaklines=true,                                     % line wrapping on
%  language=ocl,
%  frame=ltrb,
%  framesep=5pt,
%  basicstyle=\normalsize,
%  keywordstyle=\ttfamily\color{OliveGreen},
%  identifierstyle=\ttfamily\color{CadetBlue}\bfseries,
%  commentstyle=\color{Brown},
%  stringstyle=\ttfamily,
%  showstringspaces=ture
%}

\addtolength{\topmargin}{-50pt} 
\addtolength{\textwidth}{145pt}
\addtolength{\textheight}{120pt}
\addtolength{\oddsidemargin}{-70pt}

%\newcommand{\minix}{\textsl{minix }}

%%% Encabezado y pie de p'agina
\pagestyle{fancy}
\fancyhead[LO]{Base de Datos}
\fancyhead[C]{}
\fancyhead[RO]{P\'agina \thepage\ de \pageref{LastPage}}
\renewcommand{\headrulewidth}{0.4pt}
\fancyfoot{}

\newcommand{\FALTA}{{\bf FALTA FALTA FALTA FALTA FALTA FALTA FALTA FALTA }}
\def\var#1{\textsl{#1}}

\begin{document}

\universidad{Universidad de Buenos Aires}
\facultad{Facultad de ciencias exactas y naturales}
\departamento{Departamento de Computacion}
\materia{Base de Datos}
\resumen{Tp1}
\keys{MER, MR}
\titulo{Tp1}
\subtitulo{Tp 1: Empresa de Transportes}
\grupo{Numero de grupo: 1}
\fecha{2do Cuatrimeste 2011}
\footspace{1cm}
\integrante{Piotrkowski, Kevin}{204/06}{marcoskevin@gmail.com}
\integrante{Rugnone, Mart�n}{665/04}{mrugnone@hotmail.com}
\integrante{Alvarez, Mar�a de los Angeles}{264/05}{mdelosaalvarez@hotmail.com}
\integrante{Engler, Christian}{314/05}{caeycae@gmail.com}

\maketitle{}

\tableofcontents

\newpage

\section{Hip�tesis}
\begin{enumerate}
	\item Una ruta es un camino posible para ir desde un punto a otro
	\item Ruta es d�bil en relaci�n a un recorrido porque s�lo sirve como una forma de ir desde el origen del recorrido a su destino
	\item Contingencia es una entidad d�bil porque puede haber una cantidad arbitraria de ellas por cada viaje realizado(en contraposici�n a una lista de contingencias predefinidas ej:choque, bloqueo de ruta, control policial, etc) y las mismas no tienen sentido sin su viaje relacionado
	\item Consideramos que estado del veh�culo se refiere a una descripci�n general del mismo(ej:buen estado, averiado, abollado,...) en vez de referirse al discriminante que separa a los veh�culos entre ``en uso'' y ``en reparaci�n''
\end{enumerate}

\section{MER}

\FALTA GRAFICO

\subsection{Restricciones}

\begin{enumerate}
	\item Para todo recorrido \var{r1}, no existe recorrido \var{r2} tal que r2 es distinto de \var{r1}, el destino de \var{r2} es distinto al de \var{r1} y el origen de \var{r2} es distinto al de \var{r1}.
	\item Para todo recorrido, el origen no puede ser igual al destino
	\item El recorrido de la ruta de todo viaje realizado debe ser igual al recorrido de ese viaje.
	\item Para todo recorrido que tiene un viaje asociado, existe una ruta.
	\item Para todo viaje la cantidad de conductores debe ser entre 1 y 3 y la fechaHoraLlegadaEst debe ser mayor a fechaHoraPartida.
	\item Para todo viaje realizado la fechaHoraLlegada debe ser mayor a fechaHoraPartida.
	\item Para todo viaje planificado el vehiculo utilizado debe estar en uso.	
	\item Para todo vehiculo en reparacion la fecha de ingreso es mayor a la fecha de alta.
	\item Para toda licencia, la fecha de obtencion es menor a la fecha de vencimiento
	\item Para todo chofer, la fecha de nacimiento es menor que la fecha de obtenci�n de su licencia
	\item Para toda ruta, hay un �nico clima para todo per�odo definido
	\item Para todo control, su fecha de realizaci�n debe ser anterior a la fecha de partida del viaje que se est� controlando  
	\item Consideramos per�odo como estaci�n del a�o, osea: verano, invierno, oto�o y primavera.
\end{enumerate}


\section{MR}

\begin{mr}{Agencia de Viajes}

\entidad{Licencia}{\pk{nroLicencia}, fechaObtencion, fechaVencimiento, observaciones}
\entidadPK{nroLicencia}
\entidadCK{nroLicencia}

\entidad{Chofer}{\pk{nroDocumento}, fechaNac, nomApe, domicilio, telefono, \fk{nroLicencia}}
\entidadPK{nroDocumento}
\entidadCK{nroDocumento}
\entidadFK{nroLicencia}

\entidad{Conduce}{\pfk{nroDocumento}, \pfk{codViaje}}
\entidadPK{(nroDocumento, codViaje)}
\entidadCK{(nroDocumento, codViaje)}
\entidadFK{nroDocumento, codViaje}

\entidad{Control}{\pk{codControl}, \fk{nroDocumento}, \fk{codViaje}, \fk{codTipo}, resultadoTest, fechaControl}
\entidadPK{codControl}
\entidadCK{codControl}
\entidadFK{(nroDocumento, codViaje), codTipo}

\entidad{TipoTest}{\pk{codTipo}, descripcion}
\entidadPK{codTipo}
\entidadCK{codTipo}

\entidad{Contingencia}{\pk{codContingencia}, \pfk{codViaje}, descripcion}
\entidadPK{(codContingencia, codViaje)}
\entidadCK{(codContingencia, codViaje)}
\entidadFK{codViaje}

\entidad{Direccion}{\pk{calle}, \pk{altura}, \pfk{codCiudad}}
\entidadPK{(calle, altura, codCiudad)}
\entidadCK{(calle, altura, codCiudad)}
\entidadFK{codCiudad}

\entidad{Ciudad}{\pk{codCiudad}, nombre}
\entidadPK{codCiudad}
\entidadCK{codCiudad}

\entidad{Recorrido}{\pk{codRecorrido}, nombre, \fk{calleOrig}, \fk{alturaOrig}, \fk{codCiudadOrig}, \fk{calleDest}, \fk{alturaDest}, \fk{codCiudadDest}}
\entidadPK{codRecorrido}
\entidadCK{codRecorrido}
\entidadFK{(calleOrig, alturaOrig, codCiudadOrig), (calleDest, alturaDest, codCiudadDest)}

\entidad{Ruta}{\pk{codRuta}, \pfk{codRecorrido}, cantKm, condicionesCamino, cantPeajes, tiempoEstimado}
\entidadPK{(codRuta, codRecorrido)}
\entidadCK{(codRuta, codRecorrido)}
\entidadFK{codRecorrido}

\entidad{Estado}{\pk{codEstado}, descripcion}
\entidadPK{codEstado}
\entidadCK{codEstado}

\entidad{Vehiculo}{\pk{nroPatente}, modelo, marca, capacidad, fechaAlta, \fk{codEstado}, enUso, fechaIngresoReparacion}
\entidadPK{nroPatente}
\entidadCK{nroPatente}
\entidadFK{codEstado}

\entidad{Viaje}{\pk{codViaje}, fechaHoraPartida, fechaHoraLlegadaEst, \fk{nroPatente}, \fk{codRecorrido}, realizado}
\entidadPK{codViaje}
\entidadCK{codViaje}
\entidadFK{nroPatente,codRecorrido}

\entidad{ViajeRealizado}{\pk{codViaje}, fechaHoraLlegada, \fk{codRuta}, \fk{codRutaRecorrido}}
\entidadPK{codViaje}
\entidadCK{codViaje}
\entidadFK{codViaje, (codRuta, codRutaRecorrido)}

\entidad{Participa}{\pfk{codRecorrido}, \pfk{codRuta}, \pfk{nombreEstacion}, \fk{codClima}}
\entidadPK{((codRecorrido, codRuta), nombreEstacion)}
\entidadCK{((codRecorrido, codRuta), nombreEstacion)}
\entidadFK{(codRecorrido, codRuta), nombreEstacion, codClima}

\entidad{Estacion}{\pk{nombreEstacion}}
\entidadPK{nombreEstacion}
\entidadCK{nombreEstacion}

\entidad{Clima}{\pk{codClima}, descripcion}
\entidadPK{codClima}
\entidadCK{codClima}

\paragraph{Nota: a menos que expl�citamente se mencione que un valor debe estar en otra entidad y viceversa, no vale la vuelta}

\begin{enumerate}
	\item $Chofer.nroLicencia$ debe estar en $Licencia.nroLicencia$ y viceversa
	\item $Conduce.codViaje$ debe estar en $Viaje.codViaje$ y viceversa
	\item $Conduce.nroDocumento$ debe estar en $Chofer.nroDocumento$
	\item $Control.nroDoumento$ debe estar en $Chofer.nroDocumento$
	\item $Control.codViaje$ debe estar en $Viaje.codViaje$
	\item $Control.codTipo$ debe estar en $TipoTest.codTipo$
	\item $Contingencia.codViaje$ debe estar en $ViajeRealizado.codViaje$
	\item $Direccion.codCiudad$ debe estar en $Ciudad.codCiudad$
	\item $Recorrido.calleOrig$ debe estar en $Direccion.calle$
	\item $Recorrido.alturaOrig$ debe estar en $Direccion.altura$
	\item $Recorrido.calleDest$ debe estar en $Direccion.calle$
	\item $Recorrido.alturaDest$ debe estar en $Direccion.altura$
	\item $Ruta.codRecorrido$ debe estar en $Recorrido.codRecorrido$
	\item $Vehiculo.fechaReparacion$ debe ser NULL si y solo si $Vehiculo.enUso$ es cierto
	\item $Vehiculo.codEstado$ debe estar en $Estado.codEstado$
	\item $Viaje.nroPatente$ debe estar en $Vehiculo.nroPatente$
	\item $Viaje.codRecorrido$ debe estar en $Recorrido.codRecorrido$
	\item $ViajeRealizado.codRuta$ debe estar en $Ruta.codRuta$
	\item $ViajeRealizado.codRutaRecorrido$ debe estar en $Recorrido.codRecorrido$
	\item $Participa.codRecorrido$ debe estar en $Recorrido.codRecorrido$
	\item $Participa.nombreEstacion$ debe estar en $Estacion.nombreEstacion$
	\item $Participa.codClima$ debe estar en $Clima.codClima$
	\item $Licencia.fechaObtencion$ es anterior a $Licencia.fechaVencimiento$
	\item Para todo recorrido \var{r1}, no existe recorrido \var{r2} tal que r2 es distinto de \var{r1}, el destino de \var{r2} es distinto al de \var{r1} y el origen de \var{r2} es distinto al de \var{r1}.
	\item Para todo recorrido, el origen no puede ser igual al destino
	\item El recorrido de la ruta de todo viaje realizado debe ser igual al recorrido de ese viaje.
	\item Para todo recorrido que tiene un viaje asociado, existe una ruta.
	\item Para todo viaje la cantidad de conductores debe ser entre 1 y 3 y la fechaHoraLlegadaEst debe ser mayor a fechaHoraPartida.
	\item Para todo viaje realizado la fechaHoraLlegada debe ser mayor a fechaHoraPartida.
	\item Para todo viaje planificado el vehiculo utilizado debe estar en uso.	
	\item Para todo vehiculo en reparacion la fecha de ingreso es mayor a la fecha de alta.
	\item Para toda licencia, la fecha de obtencion es menor a la fecha de vencimiento
	\item Para todo chofer, la fecha de nacimiento es menor que la fecha de obtenci�n de su licencia
	\item Para toda ruta, hay un �nico clima para todo per�odo definido
	\item Para todo control, su fecha de realizaci�n debe ser anterior a la fecha de partida del viaje que se est� controlando
\end{enumerate}

\end{mr}


\end{document}
