\section{Conclusiones}

Con respecto al algoritmo de \fs, vemos que ninguna de las estrategias de reemplazo de p�ginas es mejor a otra.
\newline
Pero uno puede observar resultados interesantes cuando se necesita recorrer m�s de una vez la traza, dado que ahora hay oportunidad de cachear p�ginas pedidas anteriormente.
\newline
Para el caso de dos lecturas sucesivas sobre la misma traza podemos considerar dos escenarios:
\begin{itemize}
 \item	\textbf{La cantidad de frames en buffer es menor a la cantidad de p�ginas pedidas}: la �nica estrategia que tiene Hit-Rate mayor a 0 es MRU. FIFO y LRU cuando necesitan alojar una nueva p�gina empiezan desalojando la primera en ser almacenada y este comportamiento impide reaprovechar esas p�ginas en la segunda lectura de la traza.
 \item	\textbf{La cantidad de frames en buffer en mayor o igual a la cantidad de p�ginas pedidas}: en este escenario las tres estrategias muestran un HitRate del 50\% dado que si bien en la primera lectura no tienen ninguna p�gina en buffer, para la segunda no tuvieron que desalojar ninguna(son todos hits).
\end{itemize}

En cuanto al algoritmo BNLJ, podemos considerar los siguientes escenarios:
\begin{itemize}
  \item Las p�ginas de R pueden ser cacheadas en memoria en su totalidad dejando por lo menos dos frames de memoria libres,uno para un bloque de S y otro para el resultado del JOIN. Este caso no es muy interesante, ya que con cualquier algoritmo de reemplazo el hit-rate es cero. Esto se da porque tengemos cacheado R completo y lo que necesitamos ir iterando es el bloque de S contra el cu�l cruzamos los bloques de R. Es decir, en la primer iteraci�n cruzamos todos los bloques de R contra el primero de S y obtenemos un miss, luego paso a cargar en memoria en segundo bloque de S y en caso de tener que reemplazar el �nico bloque que hizo release fue el bloque anterior de S y volvemos a obtener otro miss, y as� sucesivamente hasta iterar sobre todo S.
  \item La totalidad de las p�ginas de R no pueden ser cacheadas en memoria y adem�s tenemos dos frames de memoria libres, uno para un bloque de S y otro para el resultado del JOIN. 
  \item TODO TODO TODO TODO TODO TODO TODO TODO TODO TODO TODO TODO TODO TODO TODO TODO TODO TODO TODO TODO 
\end{itemize}

En cuanto al INLJ con un �ndice B+ Clustered,
TODO TODO TODO TODO TODO TODO TODO TODO TODO TODO TODO TODO TODO TODO TODO TODO TODO TODO TODO TODO.

En cuanto al INLJ con un �ndice B+ UnClustered, 
  TODO TODO TODO TODO TODO TODO TODO TODO TODO TODO TODO TODO TODO TODO TODO TODO TODO TODO TODO TODO 
