
\section{Implementaci�n de estrategias}

\subsection{Implementaci�n de LRU y MRU}

En el paquete \textbf{ubadb.components.bufferManager.bufferPool.replacementStrategies.lru} creamos las \\clases: 
\begin{itemize}
	\item LRUBufferFrame \\ Esta clase extiende BufferFrame para mantener para cada frame un nroAcceso secuencial, para luego poder en el strategy decidir el m�s antiguo.
	\item LRUReplacementStrategy \\ Implementaci�n del algoritmo LRU
\end{itemize}

En el paquete \textbf{ubadb.components.bufferManager.bufferPool.replacementStrategies.lru} del directorio src/test creamos los siguientes test de unidad: 
\begin{itemize}
	\item LRUBufferFrameTest \\ Verifica dada una secuencia de creaciones, pins y unpins, que los contadores queden seteados correctamente.
	\item LRUReplacementStrategyTest \\ Verifica que los frames seleccionados por el algoritmo se den seg�n lo esperado por distintas trazas determinadas.
\end{itemize}

Todo lo comentado se hace tambi�n para la pol�tica de reemplazo MRU en el paquete \\ \textbf{ubadb.components.bufferManager.bufferPool.replacementStrategies.mru}

\subsection{Implementaci�n extras}

\begin{itemize}
	\item Creamos la clase \textbf{ubadb.apps.bufferManagement.PageReferenceTraceReader} para poder parsear trazas como las dadas de ejemplo por la c�tedra.
	\item Creamos la clase \textbf{ubadb.apps.bufferManagement.DemoStrategy} para ejecutar todas las trazas de un directorio variando la pol�tica de reemplazo de p�ginas y el tama�o del SingleBuffer.
	\item Sobrescribimos el m�todo \verb|toString()| de \textsl{SingleBufferPool} para poder imprimirlo por pantalla y estudiar la disposici�n del buffer.
\end{itemize}

\subsection{Modificaciones en SingleBufferPool}
\begin{itemize}
	\item Implementamos el m�todo toString() para imprimir el estado del buffer con las p�ginas actualmente almacenadas en el mismo.
	\item Asociamos un orden a cada frame almacenado para poder imprimir el buffer con los frames seg�n el orden en que fueron siendo agregados.
\end{itemize}

\subsection{Ejecuci�n del c�digo}
Pasos para la correcta ejecuci�n del programa desde eclipse:
\begin{itemize}
	\item Configurar en Run $>$ Run Configurations $>$ Arguments $>$ Working Directory: la carpeta TARGET correspondiente al c�digo.
	\item Ir a la clase ubadb $>$  apps $>$  bufferManagement $>$ DemoStrategy y correr el proyecto desde esta clase.
\end{itemize}	


Pasos para la correcta ejecuci�n del programa desde consola. Ejecutar los siguientes comandos:
\begin{itemize}
	\item 
			\begin{verbatim}
				mvn clean
			\end{verbatim}
	\item 
			\begin{verbatim}
				mvn install
			\end{verbatim}
		\item 
			\begin{verbatim}
				cd target
			\end{verbatim}
		\item 
			\begin{verbatim}
				java -jar ubadb-1.0.jar
			\end{verbatim}
\end{itemize}	
