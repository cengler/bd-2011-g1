\section{Dise�o F�sico}
\subsection{Aclaraciones}
\begin{itemize}
	\item Vehiculo en reparaci�n no se pone con un esquema relaci�n sino que se pone la restricci�n del tipo de vehiculo en funci�n de la fecha de ingreso a reparaci�n. 
	\item En la tabla \textbf{control} el campo \textit{resultadoTest} es de tipo bit ya que los test pueden estar aprobados o desaprobador. Es decir toma los valores $0$ y $1$, donde $1$ significa aprobado y $0$ significa desaprobado.
	\item Los vehiculos pueden estar \textbf{en uso} o \textbf{en reparaci�n}. En la tabla \textbf{vehiculo} se tiene un campo llamado \textit{EnUso}. Este campo es de tipo bit ya que toma solo dos valores: $1$ significa en uso y $0$ significa en reparaci�n.
	\item En todos los campos de tipo char, estos se tomaron de longitud 30.   
\end{itemize}

\subsection{Implementaci�n}

Para el presente trabajo se decidi� utilizar el motor de base de datos SQL Server 2005. El mismo se encuentra disponible en las computadores de los laboratorios del departamento de computaci�n por lo que en los mismos se podr� regenerar los scripts de creaci�n, inserci�n y consultas entregados en el cd adjunto a este documento.

\subsubsection{Consultas}
// tipear con negrita las consultas y decir como se resolvieron

\begin{enumerate}
\item \textbf{Los recorridos para los cuales se usaron todas las rutas posibles registradas para ese recorrido, para viajes realizados el a�o pasado y recorridos asociados a m�s de una ruta}
\item \textbf{El promedio de los viajes realizados por veh�culo por a�o y el estado en que �ste se encuentra}
\item \textbf{Los choferes que han utilizado todos los veh�culos de menos de dos a�os de antig�edad, en viajes del mismo semestre. }
\end{enumerate}



\subsubsection{Triggers}
// decidir que triggers se implementaron y c�mo se hicieron 

\begin{itemize}
\item \FALTA
\end{itemize}
